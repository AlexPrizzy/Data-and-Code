% Options for packages loaded elsewhere
\PassOptionsToPackage{unicode}{hyperref}
\PassOptionsToPackage{hyphens}{url}
%
\documentclass[
]{article}
\usepackage{lmodern}
\usepackage{amssymb,amsmath}
\usepackage{ifxetex,ifluatex}
\ifnum 0\ifxetex 1\fi\ifluatex 1\fi=0 % if pdftex
  \usepackage[T1]{fontenc}
  \usepackage[utf8]{inputenc}
  \usepackage{textcomp} % provide euro and other symbols
\else % if luatex or xetex
  \usepackage{unicode-math}
  \defaultfontfeatures{Scale=MatchLowercase}
  \defaultfontfeatures[\rmfamily]{Ligatures=TeX,Scale=1}
\fi
% Use upquote if available, for straight quotes in verbatim environments
\IfFileExists{upquote.sty}{\usepackage{upquote}}{}
\IfFileExists{microtype.sty}{% use microtype if available
  \usepackage[]{microtype}
  \UseMicrotypeSet[protrusion]{basicmath} % disable protrusion for tt fonts
}{}
\makeatletter
\@ifundefined{KOMAClassName}{% if non-KOMA class
  \IfFileExists{parskip.sty}{%
    \usepackage{parskip}
  }{% else
    \setlength{\parindent}{0pt}
    \setlength{\parskip}{6pt plus 2pt minus 1pt}}
}{% if KOMA class
  \KOMAoptions{parskip=half}}
\makeatother
\usepackage{xcolor}
\IfFileExists{xurl.sty}{\usepackage{xurl}}{} % add URL line breaks if available
\IfFileExists{bookmark.sty}{\usepackage{bookmark}}{\usepackage{hyperref}}
\hypersetup{
  hidelinks,
  pdfcreator={LaTeX via pandoc}}
\urlstyle{same} % disable monospaced font for URLs
\usepackage[margin=1in]{geometry}
\usepackage{graphicx,grffile}
\makeatletter
\def\maxwidth{\ifdim\Gin@nat@width>\linewidth\linewidth\else\Gin@nat@width\fi}
\def\maxheight{\ifdim\Gin@nat@height>\textheight\textheight\else\Gin@nat@height\fi}
\makeatother
% Scale images if necessary, so that they will not overflow the page
% margins by default, and it is still possible to overwrite the defaults
% using explicit options in \includegraphics[width, height, ...]{}
\setkeys{Gin}{width=\maxwidth,height=\maxheight,keepaspectratio}
% Set default figure placement to htbp
\makeatletter
\def\fps@figure{htbp}
\makeatother
\setlength{\emergencystretch}{3em} % prevent overfull lines
\providecommand{\tightlist}{%
  \setlength{\itemsep}{0pt}\setlength{\parskip}{0pt}}
\setcounter{secnumdepth}{-\maxdimen} % remove section numbering

\author{}
\date{\vspace{-2.5em}}

\begin{document}

title: Jan 19th in class problems author: Alex Przybycin

With the data loaded, answer the following questions. The objective here
is twofold: 1) to practice your statistical computing skills, and 2)
apply and explore error from fitting models on different sets of data.

\begin{enumerate}
\def\labelenumi{\arabic{enumi}.}
\item
  Using some of the techniques we covered last week:

  \begin{enumerate}
  \def\labelenumii{\alph{enumii}.}
  \item
    Select only the Obama feeling thermometer (\texttt{ftobama}),
    household income (\texttt{faminc}), party affiliation on a 3 point
    scale (\texttt{pid3}), birth year (\texttt{birthyr}), and gender
    (\texttt{gender}) (\emph{be sure to recode missing values to
    \texttt{NA} and omit these})
  \item
    Split the subset data into training (75\%) and testing (25\%) sets
    (\emph{hint}: remember to set the seed (\texttt{set.seed()}) prior
    to creating the split, as the proportions are generated at random)
  \item
    Plot the distributions of each against each other to ensure they
    look similar
  \end{enumerate}
\end{enumerate}

library(here) library(tidyverse) NESdta \textless-
read\_csv(here(``data'', ``anes\_2016.csv''))

NESdta\_short \textless- NESdta \%\textgreater\% select(ftobama, pid3,
birthyr, gender, faminc) head(NESdta\_short, n = 5)

set.seed(1234)

split\_tidy \textless- initial\_split(NESdta\_short, prop = 0.75)

train\_tidy \textless- training(split\_tidy) test\_tidy \textless-
testing(split\_tidy)

tidy \textless- train\_tidy \%\textgreater\% ggplot(aes(x = ftobama)) +
geom\_line(stat = ``density'') + geom\_line(data = test\_tidy, stat =
``density'', col = ``red'') + labs(title = ``Via Tidymodels'') +
theme\_minimal() ```

\begin{enumerate}
\def\labelenumi{\arabic{enumi}.}
\setcounter{enumi}{1}
\tightlist
\item
  Fit a linear regression (\texttt{lm()}) on the \emph{training} data,
  predicting trump approval as a function of all other features.
\end{enumerate}

training\_regression \textless- lm( ftobama \textasciitilde{}
pid3+birthyr+gender+faminc, data = train\_tidy)

\begin{enumerate}
\def\labelenumi{\arabic{enumi}.}
\setcounter{enumi}{2}
\tightlist
\item
  Calculate the training mean squared error (\emph{hint}: consider using
  the \texttt{mse()} function from Dr.~Soltoff's \texttt{rcfss} package,
  which is at the uc-cfss github, \emph{not} on CRAN).
\end{enumerate}

library(rcfss)

mse(train\_tidy)

\begin{enumerate}
\def\labelenumi{\arabic{enumi}.}
\setcounter{enumi}{3}
\tightlist
\item
  Calculate predictions for the testing set, using the model you built
  on the training set (\emph{hint}: consider either \texttt{predict()}
  from base R, or \texttt{augment()} from \texttt{broom}).
\end{enumerate}

library(base)

test\_predict = predict(train\_tidy)

\begin{enumerate}
\def\labelenumi{\arabic{enumi}.}
\setcounter{enumi}{4}
\tightlist
\item
  Calculate the testing mean squared error.
\end{enumerate}

training\_mse = mean(train\_tidy\$residuals\^{}2)

\begin{enumerate}
\def\labelenumi{\arabic{enumi}.}
\setcounter{enumi}{5}
\tightlist
\item
  Compare the mean squared error from both sets numerically,
  side-by-side. What do you see? \emph{Discuss in your groups and record
  a few sentences as a response.}
\end{enumerate}

I'm not sure as to how to do this numerically. The function I used about
gets the mean of the sum of squares from the training data.

\begin{enumerate}
\def\labelenumi{\arabic{enumi}.}
\setcounter{enumi}{6}
\tightlist
\item
  Write your own function to calculate the MSE. Then, use it to
  re-answer questions 3 and 5. Present the results here, and compare
  with the \texttt{rcfss} approach via \texttt{mse()}. These results
  should be identical to the \texttt{mse()} version. Are they? If not,
  \emph{why} do you think? (\emph{just a sentence or two on your general
  thoughts if they differ})
\end{enumerate}

I couldn't find the mse() function on the cfss github. Am I supposed to
install that as a package in R first and find the documentation for it
through R?

\end{document}
